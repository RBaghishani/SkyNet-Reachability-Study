\section{Experimental Analysis}

\subsection{Experimental Setup}

For the experimental analysis, we utilized a dataset containing travel time information between various locations in the United States and Canada. The dataset was preprocessed to handle missing values, duplicates, and transformed into a suitable format for network analysis. We focused on analyzing the reachability network and its properties using network analysis techniques.

\subsection{Reachability Network Analysis}

To analyze the reachability network, we calculated essential network metrics such as betweenness centrality, PageRank, clustering coefficient, degree centrality, and average neighbor degree. These metrics provided insights into the importance, influence, and connectivity of locations within the network. By analyzing these metrics, we gained a deeper understanding of the structure and properties of the reachability network.

\subsection{Impact of Travel Time Thresholds}

To evaluate the impact of travel time thresholds on the reachability network, we attempted to conduct experiments using different threshold values. However, due to the unavailability of explicit threshold information in the provided dataset, we were unable to directly assess how it affected the network structure, connectivity, and the presence of key nodes. Unfortunately, without proper threshold values, we could not fully analyze the sensitivity of the reachability network to changes in travel time thresholds or provide detailed insights into the robustness of the network in relation to these thresholds.

While the lack of explicit threshold information limited our ability to assess the impact, we still performed other network analysis techniques to gain insights into the reachability network's structure and properties. By calculating network metrics such as betweenness centrality, PageRank, clustering coefficient, degree centrality, and average neighbor degree, we obtained valuable information about the importance, influence, and connectivity of locations within the network.

Although we were unable to directly analyze the impact of travel time thresholds, we believe that the analysis of these network metrics provided valuable insights into the reachability network's characteristics and dynamics. We acknowledge that future research efforts with access to proper threshold information could further investigate the relationship between travel time thresholds and the reachability network's properties.

\subsection{Comparison of US and Canada Networks}

We also compared the reachability networks of the United States and Canada to understand their similarities and differences. By analyzing network metrics, such as degree centrality, betweenness centrality, and clustering coefficient, we assessed the variations in the structure and connectivity patterns between the two networks. This comparison allowed us to identify any unique characteristics or properties specific to each country's reachability network.

\subsection{Visualization of Results}

To enhance the understanding of the experimental analysis, we created visualizations of the reachability networks and their properties. Network graphs were generated to visually represent the nodes (locations) and edges (travel connections) within the network. Additionally, scatter plots and heatmaps were utilized to visualize the relationships between different network metrics and travel time thresholds. These visualizations aided in interpreting the results and identifying any patterns or trends.


\subsection{Interpretation and Discussion}

Based on the experimental analysis, we interpreted and discussed the findings in the context of the research objectives. We identified key locations with high centrality measures, influential nodes, and communities within the reachability networks. We also discussed the impact of travel time thresholds on network connectivity and highlighted any notable differences in the reachability networks of the United States and Canada. Additionally, we acknowledged any limitations of the analysis and suggested potential areas for future research.