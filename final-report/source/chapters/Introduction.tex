\section{Introduction}

\subsection{Motivation}

Air travel has become increasingly important in modern society, connecting people and facilitating economic growth. The development of complex transportation accessibility networks has played a crucial role in shaping the efficiency and connectivity of air travel systems. Understanding the underlying structures and characteristics of these networks is essential for optimizing their design, improving connectivity, and enhancing overall transportation accessibility.

\subsection{Objectives}

The objective of this project is to analyze and compare the reachability networks of cities in the United States and Canada. By studying the network structures, node characteristics, and travel time, we aim to gain insights into the connectivity and accessibility of these networks. This analysis will help us understand the impact of network structures, node properties, and travel time on the overall connectivity and efficiency of the airline travel reachability networks.

\subsection{Dataset Overview}

The dataset used in this project is a transportation reachability network of cities in the United States and Canada. The dataset includes information on the population of metropolitan cities, latitude, and longitude, allowing us to study the relationship between network properties and city characteristics. The edges of the graph are weighted based on estimated travel time, including stopover delays, providing a comprehensive representation of the airline travel reachability network. With 456 nodes and 71,959 edges, this dataset offers a rich source of information for our analysis.

By conducting a comparative analysis of the reachability networks of the United States and Canada, we aim to identify similarities and differences between the two networks. This analysis will provide valuable insights into the structural properties, node characteristics, and travel time impact on network connectivity. The findings of this study can contribute to the optimization and improvement of airline travel systems, leading to enhanced connectivity and accessibility for passengers.